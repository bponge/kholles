\documentclass[french,12pt,a4paper]{article}

\newcommand*{\texpath}{../tex}%
\input{\texpath/preamble}


\author{Olivier Reynet}
\title{Colle de mathématiques}
\date{\today}


\begin{document}

\maketitle

\begin{exercise}[subtitle= Propriétés de la fonction partie entière]
	\begin{tasks}
		\task \Mq $\forall x \in \RR, \forall n \in \NN, \lfloor x + n \rfloor = \lfloor x \rfloor +n$.
		\task \Mq $\exists x \in \RR, \exists n \in \NN, \lfloor nx \rfloor \neq n \lfloor x \rfloor$. Conclure en prenant la négation. 
	\end{tasks}
\end{exercise}


\begin{exercise}[subtitle= Étude d'une fonction]
	Pour la fonction $f$ suivante, on suivra le protocole standard pour étudier une fonction, à savoir :
	\begin{enumerate}
		\item Déterminer l'ensemble de définition $\Dc_f$.
		\item Déterminer les symétries éventuelles de $f$ : parité et périodicité.
		\item Déterminer la dérivabilité de $f$,  zéros et infinis de $f$ pour les tangentes.
		\item Compléter un tableau de variation de $f$ qui comporte : les bornes de $\Dc_f$, les coupures dans $\Dc_f$ où $f$ n'est pas définie, les points clefs de $f'$, le signe de $f'$ et les variations de $f$ avec les valeurs limites au bout des flèches.
		\item Déterminer quelques points marquant de $f$, les zéros par exemple.
		\item Étudier les asymptotes de $f$.
		\item Tracer l'allure du graphe de $f$ avec tous les éléments.
	\end{enumerate}

	$f(x)= \ln{\abs{x}}+x^3$

\end{exercise}

	\begin{exercise}[subtitle= Dé à 20 faces]
		On dispose de 3 dés identiques à vingt faces et on les lance simultanément dans le but de disposer de 3 notes de colle. 
		\begin{tasks}
			\task Combien y-a-t-il de lancers possibles ?
			\task Combien y-a-t-il de lancers possibles tels que tous les dés présentent une valeur supérieure ou égale à 10 ?
			\task Combien y-a-t-il de lancers possibles tels que deux dés exactement présentent une valeur identique ? 
			\task Combien y-a-t-il de lancers possibles tels que les dés présentent  les valeurs 17, 9 et 3 ?
		\end{tasks}
	\end{exercise}

\end{document}



