 \chapter{Fonctions}
 
\section{Programme}
	\begin{itemize}
 	\item Chapitre précédent
 	\item Ensemble de définition, courbe représentative,
 	\item Parité, périodicité, monotonie,
 	\item Fonction majorée, minorée, bornée, extrema et extrema locaux,
 	\item Fonctions usuelles : trinôme, racines, puissances, ln, exp, cos, sin, valeur absolue, puissance d'un réel et d'une fonction de réels. 
 \end{itemize}




\section{Remarques}

	\NB Bien faire la différence entre majorant, minorant, bornes inférieure et supérieure et extrema. 
	
	


\section{Questions de cours}	


\begin{exercise}[subtitle=Parité et périodicité, extype=cours]
	\begin{tasks} 
		\task Définir la parité d'une fonction sur un ensemble de définition.
		\task Quel lien peut-on faire entre la symétrie d'une courbe et la nature de sa parité ?
		\task Donner un exemple de fonction pour chaque type de parité.
		\task Définir le concept de fonction périodique. 
		\task Donner un exemple de fonction périodique.
		\task Quel lien peut-on faire entre la translation d'une courbe et la périodicité ?
	\end{tasks}
\end{exercise}

\begin{exercise}[subtitle=Monotonie, extype=cours]
	\begin{tasks} 
		\task Quand peut-on qualifier une fonction $f$ de (strictement) monotone ?
		\task Donner un exemple de fonction monotone.
		\task Donner la définition d'une fonction strictement croissante sur I.
		\task Existe-t-il des fonctions croissantes et décroissantes sur I simultanément ?
		\task La somme de deux fonctions croissantes sur I est-elle croissante sur I ? Peut-on le démontrer ?
	\end{tasks}
\end{exercise}

\begin{exercise}[subtitle=Majorants et minorants, extype=cours]
	\begin{tasks} 
		\task Définir le concept de fonction bornée.
		\task Donner un exemple de fonction bornée.
		\task Donner la définition d'un maximum et d'un maximum local.
		\task Une fonction bornée possède-t-elle toujours un maximum et/ou un minimum ? Donner un exemple. 
	\end{tasks}
\end{exercise}



\section{Exercices}



\begin{exercise}[subtitle= Résolution d'équations, difficulty=*]
	Résoudre dans $\RR$ les équations suivantes :
		\begin{tasks}
		\task $(E): \ln(x^2-1)+\ln(4)=\ln(4x-1)$
		\task $(E): 2^{x^2}=3^{x^3}$
		\task $(E): \ln \abs{x-1} + \ln \abs{x-2} = \ln \abs{4x^2 + 3x-7}$
		\task $(E): x^{\sqrt{x}}=\sqrt{x}^x$
	    \task $(E): 2^{x+1}+4^x = 15$
		\task $(E): 4^x - 3^{x-\frac{1}{2}} = 3^{x+\frac{1}{2}} -2^{2x-1}$ 
		\task $(E): \frac{\ln x}{\ln a} =\frac{\ln a}{\ln x}$ où $a\in ]0,1[\cup]1, +\infty[$
		\task $(E) : \sqrt{x} + \sqrt[3]{x}=2$
		\end{tasks}
\end{exercise}

\begin{exercise}[subtitle= Étude de fonctions, difficulty=*]
Pour les fonctions $f$ suivantes, on suivra le protocole standard pour étudier une fonction, à savoir :
\begin{enumerate}
	\item Déterminer l'ensemble de définition $\Dc_f$.
	\item Déterminer les symétries éventuelles de $f$ : parité et périodicité.
	\item Déterminer la dérivabilité de $f$,  zéros et infinis de $f$ pour les tangentes.
	\item Compléter un tableau de variation de $f$ qui comporte : les bornes de $\Dc_f$, les coupures dans $\Dc_f$ où $f$ n'est pas définie, les points clefs de $f'$, le signe de $f'$ et les variations de $f$ avec les valeurs limites au bout des flèches.
	\item Déterminer quelques points marquant de $f$, les zéros par exemple.
	\item Étudier les asymptotes de $f$.
	\item Tracer l'allure du graphe de $f$ avec tous les éléments.
\end{enumerate}
	\begin{tasks}(2)
		\task $f(x)= \frac{1}{1-e^x}$
	    \task $f(x)= \frac{e^x}{1-e^x}$
		\task $f(x)= \frac{1}{\ln{1+x}}$
		\task $f(x)= e^{\sqrt{x+1}}$
	    \task $f(x)= \ln{\sqrt{x+1}}$
	    \task $f(x)= \sin(3 \pi x)$
	    \task $f(x)= \frac{1}{\frac{1}{2}-\cos(\pi x)}$
	    \task $f(x)= \frac{cos(\pi x)}{\frac{1}{2}-\cos(\pi x)}$
	\end{tasks}
\end{exercise}

\begin{exercise}[subtitle= Résolution d'équations, difficulty=*]
	Résoudre dans $\RR$ les équations suivantes :
	\begin{tasks}
		\task $(E): \ln(x^2-1)+\ln(4)=\ln(4x-1)$
		\task $(E): 2^{x^2}=3^{x^3}$
		\task $(E): \ln \abs{x-1} + \ln \abs{x-2} = \ln \abs{4x^2 + 3x-7}$
		\task $(E): x^{\sqrt{x}}=\sqrt{x}^x$
		\task $(E): 2^{x+1}+4^x = 15$
		\task $(E): 4^x - 3^{x-\frac{1}{2}} = 3^{x+\frac{1}{2}} -2^{2x-1}$ 
		\task $(E): \frac{\ln x}{\ln a} =\frac{\ln a}{\ln x}$ où $a\in ]0,1[\cup]1, +\infty[$
		\task $(E) : \sqrt{x} + \sqrt[3]{x}=2$
	\end{tasks}
\end{exercise}

\begin{exercise}[subtitle= Extrema de fonctions, difficulty=*]
	Pour chacun de ces ensembles, montrer :
	\begin{itemize}
		\item s'il est minoré, majoré,
		\item s'il possède une borne inférieure ou supérieure, 
		\item  s'il possède un extremum.
	\end{itemize}
	\begin{tasks}(2)
		\task $\NN$
		\task $\ZZ$
		\task $E = [0,1[$
		\task $E = \{ x \in \RR, \exists (n,m) \in \NNs \times \NNs : x=\frac{1}{n} + \frac{1}{m} \}$
		\task $E = \{(-1)^n(1-\frac{1}{n}), n \in \NNs \}$
		\task $E = \{sin(\frac{n \pi}{3}), n \in \NNs \}$
		\task $E = [-1,+\infty[$
		\task $E = \{x^2 - 2x -1 <0, x \in \RR \}$
		\task $E = \{(x-1)^2-2, x \in \RR \}$
	    \task $E = \{sin(\frac{\pi}{x}), x \in ]0,2] \}$
	    \task $E = \{\frac{1}{1+x^2}, x \in [-1,1] \}$
	\end{tasks}
\end{exercise}




