 \chapter{Nombres complexes et trigonométrie}
 
\section{Programme}
	\begin{itemize}
	\item Nombres complexes, opérations, interprétation géométrique.
 	\item Conjugué et module,
 	\item Trigonométrie : cos, sin et tan, formules d’addition, valeurs remarquables, résolution d’équations trigonométriques.
 	\item Notation exponentielle : forme trigonométrique d’un nombre complexe.
 	\item  Condition d’égalité. 
 	\item Argument.
 	\item  Formules de Moivre et d’Euler.
 \end{itemize}

\section{Démonstrations à connaître}
	\begin{itemize}
		\item inégalité triangulaire,
		\item formule tan(a + b) =
		\item formule de Moivre par récurrence
\end{itemize}



\section{Remarques}

	
	
	


\section{Questions de cours}	


\begin{exercise}[subtitle=Formes d'un nombre complexe, extype=cours]
	\begin{tasks} 
		\task Définir les formes algébrique, trigonométrique et exponentielle d'un nombre complexe.
		\task Définir en particulier le nombre $i$.
		\task Comment peut-on spécifier que deux nombres complexes sont égaux ?
		\task Comment peut-on dire qu'un nombre complexe est nul ?
	\end{tasks}
\end{exercise}


\begin{exercise}[subtitle=Conjugué d'une nombre complexe, extype=cours]
	\begin{tasks} 
		\task Définir le conjugué d'un nombre complexe.
		\task Illustrer le concept de conjugué dans le plan complexe.
		\task Énoncer les relations entre un nombre complexe et son conjugué d'une part et la partie réelle et imaginaire de ce nombre d'autre part.
		\task Énoncer la relation entre un nombre complexe et son conjugué d'une part et le module de son nombre d'autre part. 
	\end{tasks}
\end{exercise}

\begin{exercise}[subtitle=Module d'un nombre complexe, extype=cours]
	\begin{tasks} 
		\task Définir le module d'un nombre complexe
		\task Illustrer graphiquement le concept de module sur le plan complexe. 
		\task Expliciter les liens entre le module d'une nombre complexe et ses différentes formes (algébrique et exponentielle).
		\task Définir l'ensemble des nombres complexe de module égal à 1. Illustrer graphiquement cet ensemble.
	\end{tasks}
\end{exercise}


\begin{exercise}[subtitle=Fonctions cosinus et sinus, extype=cours]
	\begin{tasks} 
		\task Définir quelques valeurs remarquables des fonctions sinus et cosinus, numériquement et graphiquement.
		\task Énoncer la formule fondamentale qui relie sinus et cosinus.
		\task Quelle est la parité de ces fonctions ?
		\task Quelle est la périodicité de ces fonctions ?
		\task Tracer le graphique de ces fonctions.
	\end{tasks}
\end{exercise}

\begin{exercise}[subtitle=Fonction tangente, extype=cours]
	\begin{tasks} 
		\task Définir la fonction tangente. 
		\task Quelle est la parité de cette fonction ?
		\task Quelle est la périodicité de cette fonction ?
		\task Tracer la fonction.
		\task Que vaut $tan(\frac{\pi}{2}-a)$ si l'expression est définie ?
	\end{tasks}
\end{exercise}

\begin{exercise}[subtitle=Racines nièmes, extype=cours]
	\begin{tasks} 
		\task Définir le cercle unité dans le plan complexe et sous la forme d'un ensemble.
		\task Étabir le lien entre le cercle unité et la notation exponentielle complexe.
		\task Placer $e^{i\frac{\pi}{2}}$, $e^{i\frac{5\pi}{6}}$, $e^{-i\frac{2\pi}{3}}$ sur le plan complexe.
	\end{tasks}
\end{exercise}



\section{Exercices}



\begin{exercise}[subtitle= Assertions justes ou fausses ?, difficulty=*]
		\begin{tasks} 
			\task $i$ est égal à sa partie imaginaire. 
			\task $\forall z,z' \in \CC, ( z+z' \in \RR \wedge z.z' \in \RR \Longrightarrow z,z' \in \RR)$
			\task $\forall z,z' \in \CC, (z+iz' = 0 \Longrightarrow z = z' = 0)$
			\task $\forall z \in \CC, e^z = -1 \Longrightarrow z = i\pi$
		\end{tasks}
\end{exercise}


\begin{exercise}[subtitle= Résolution d'équations dans $\RR$, difficulty=*]
	Résoudre dans $\RR$ les équations suivantes : 
	\begin{tasks}(2) 
		\task $\sqrt{3} \cos(x) - \sin(x) = \sqrt{2}$
	    \task $\cos(x) + \sqrt{3}\sin(x) = 1$
	    \task $\sqrt{2}\cos(x) + \sqrt{2}\sin(x) = -1$
	    \task $\cos(x) + \sin(x) = 1$
	\end{tasks}
\end{exercise}

\begin{exercise}[subtitle= Résolution d'équations dans $\CC$, difficulty=***]
	Résoudre dans $\CC$ les équations suivantes : 
	\begin{tasks}(2) 
		\task $4z^2 + 4z + 5 = 0$
		\task $2z^2 + 6z + 9= 0$
		\task $z^2 + z - \frac{1}{4} +\frac{1}{2}i = 0$
		\task $2z^2-(9i+1)z-7+11i=0$
		\task $z^2 - 2 e^{i\alpha} z + 2ie^{i\alpha}\sin(\alpha) = 0$
		\task $2z^3 -(1-i)z^2 + (1+i)z+2i=0$ sachant qu'elle possède une racine imaginaire pure. 
	\end{tasks}
\end{exercise}


\begin{exercise}[subtitle= Déterminer les ensembles, difficulty=**]
	\begin{tasks}(2) 
		\task $\{z \in \CC, \mid z-1 \mid = \mid z - i\mid\}$
		\task $\{z \in \CC, \mid z \mid = \mid \frac{1}{z}\mid = \mid 1 -z \mid\}$
	\end{tasks}
\end{exercise}


\begin{exercise}[subtitle=Transformer le nombre complexe sous une autre forme, difficulty=*]
	\begin{tasks}(2)
		\task $\sqrt{3} - i$ 
		\task $1 - i$
		\task (algébrique) $(\sqrt{2} - i)^4$
		\task (algébrique) $(\sqrt{2} - i)^3$
		\task $(1+i)^6$
		\task (trigonométrique)	$ (1 + i\tan{\theta})^2, \theta \in [0, \pi/2 [ $.  
	\end{tasks}
\end{exercise}



\begin{exercise}[subtitle=Sommes et racines nièmes, difficulty=*]
	Soit $n$ un entier supérieur ou égal à 2 et $\omega$ une racine nième de l'unité vérifiant $\omega^n=1$. Calculer :
	\begin{tasks}(2)
		\task  $\sum\limits_{i=0}^{n-1} w^i$
		\task $\sum\limits_{i=0}^{n-1} w^{ip}, p\in \ZZ$
		\task $\sum\limits_{i=0}^{n-1} \binom{n}{i} w^{i}$
		\end{tasks}
\end{exercise}

\begin{exercise}[subtitle=Somme de cosinus, difficulty=*]
	Calculer :
	$$S = \sum\limits_{k=0}^{5} \cos(\frac{(2k+1)\pi}{13})$$
	et 
	$$T = \sum\limits_{k=0}^{5} \sin(\frac{(2k+1)\pi}{13})$$
\end{exercise}
