 \chapter{Nombres complexes et trigonométrie}
 
\section{Programme}
	\begin{itemize}
	\item Nombres complexes, opérations, interprétation géométrique.
 	\item Conjugué et module,
 	\item Trigonométrie : cos, sin et tan, formules d’addition, valeurs remarquables, résolution d’équations trigonométriques.
 	\item Notation exponentielle : forme trigonométrique d’un nombre complexe.
 	\item  Condition d’égalité. 
 	\item Argument.
 	\item  Formules de Moivre et d’Euler.
 \end{itemize}

\section{Démonstrations à connaître}
	\begin{itemize}
		\item inégalité triangulaire,
		\item formule tan(a + b) =
		\item formule de Moivre par récurrence
\end{itemize}



\section{Remarques}

	\NB Bien faire la différence entre majorant, minorant, bornes inférieure et supérieure et extrema. 
	
	


\section{Questions de cours}	


\begin{exercise}[subtitle=Parité et périodicité, extype=cours]
	\begin{tasks} 
		\task Définir la parité d'une fonction sur un ensemble de définition.
		\task Quel lien peut-on faire entre la symétrie d'une courbe et la nature de sa parité ?
		\task Donner un exemple de fonction pour chaque type de parité.
		\task Définir le concept de fonction périodique. 
		\task Donner un exemple de fonction périodique.
		\task Quel lien peut-on faire entre la translation d'une courbe et la périodicité ?
	\end{tasks}
\end{exercise}



\section{Exercices}



\begin{exercise}[subtitle= Résolution d'équations, difficulty=*]
	Résoudre dans $\RR$ les équations suivantes :
		\begin{tasks}
		\task $(E): \ln(x^2-1)+\ln(4)=\ln(4x-1)$
		\task $(E): 2^{x^2}=3^{x^3}$
		\task $(E): \ln \abs{x-1} + \ln \abs{x-2} = \ln \abs{4x^2 + 3x-7}$
		\task $(E): x^{\sqrt{x}}=\sqrt{x}^x$
	    \task $(E): 2^{x+1}+4^x = 15$
		\task $(E): 4^x - 3^{x-\frac{1}{2}} = 3^{x+\frac{1}{2}} -2^{2x-1}$ 
		\task $(E): \frac{\ln x}{\ln a} =\frac{\ln a}{\ln x}$ où $a\in ]0,1[\cup]1, +\infty[$
		\task $(E) : \sqrt{x} + \sqrt[3]{x}=2$
		\end{tasks}
\end{exercise}
