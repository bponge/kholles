 \chapter{Calculs et raisonnements - Sommes et produits}
 
\section{Programme}
	\begin{itemize}
 	\item Calcul: symbole $\sum$, de produits $\prod$,  sommes et produits  usuels, changement d'indice, simplification téléscopique, inégalité triangulaire, résolution d’équations et d’inéquations,binôme de Newton
 	\item Raisonnement: assertions, conditions, implication et équivalence.  analyse-synthèse. Ordre des  quantificateurs et preuve d’une énoncé quantifié.
 \end{itemize}




\section{Remarques}


\NB Être très précis sur les indices !

\NB Connaître la petite relation $k\binom{n}{k}=n\binom{n-1}{k-1}$.

\NB Connaître la somme $\sum\limits_{k=0}^n \binom{n}{k}= 2^n$.

\NB Savoir appliquer la formule du binôme avec un entier $p$ et $1$ pour trouver le résultat d'une somme d'une puissance multipliée par un binôme : $\sum\limits_{k=0}^n \binom{n}{k} p^k= (p+1)^n$  

\NB Maîtriser les règles de calcul avec l'opérateur produit $\prod$. Bien les distinguer de celles de l'opérateur $\sum$. 

\section{Questions de cours}	



\begin{exercise}[subtitle=Règles de calcul associées à l'opérateur $\prod$, extype=cours]
	\begin{tasks} 
		\task Donner la définition de l'expression $ \prod\limits_{k=p}^n a_k$.
		\task Que valent les expressions $\prod\limits_{k=p}^n 1$ et  $\prod\limits_{k=p}^n a$?
		\task Définir $n!$ à l'aide de l'opérateur $\prod$. 
		\task Donner la relation de Chasles dans le cas de l'opérateur $\prod$.
		\task Que vaut l'expression $\prod\limits_{k=p}^n a_k b_k$ ?
		\task Que vaut l'expression $\prod\limits_{k=p}^n \lambda a_k$ ?
	\end{tasks}
\end{exercise}

\begin{exercise}[subtitle=Lien entre sommes et produits, extype=cours]
	\begin{tasks} 
		\task Énoncer et démontrer la règle de calcul qui lie les opérateurs $\sum$  et $\prod$  et la fonction logarithme. 
		\task Énoncer et démontrer la règle de calcul qui lie les opérateurs $\sum$  et $\prod$  et la fonction exponentielle.
	\end{tasks}
\end{exercise}


\begin{exercise}[subtitle=Simplication téléscopique, extype=cours]
	Dans cet exercice, les deux démonstrations se feront de deux manières différentes, \textit{in extenso} et par changement d'indice. 
	\begin{tasks} 
		\task Énoncer et démontrer le principe de simplification téléscopique dans le cas de la somme.
		\task  Énoncer démontrer le principe de simplification téléscopique dans le cas du produit. 
	\end{tasks}
\end{exercise}


\begin{exercise}[subtitle=Formule du binôme et preuve par récurrence, extype=cours]
	\begin{tasks} 
		\task Énoncer la formule de Pascal et l'expliquer visuellement.
		\task Énoncer la formule du binôme.
		\task Démontrer la formule par récurrence. On procèdera avec un changement d'indice et en remarquant que $\binom{n}{k-1} + \binom{n}{k}= \binom{n+1}{k}$ (formule de Pascal).
		\task Donner un exemple d'utilisation de la formule du binôme pour simplifier une somme. 
	\end{tasks}
\end{exercise}



\section{Exercices}

\begin{exercise}[subtitle= Somme et disjonction des cas, difficulty=**]
	Soit la somme $S_N = \sum\limits_{k=1}^{N}(-1)^k k$. Calculer $S_N$ en explicitant le type de raisonnement utilisé. 
\end{exercise}

\begin{exercise}[subtitle= Lien produit-somme, difficulty=*]
	Soit $x$ un nombre réel non nul et le produit $P_n = \prod\limits_{k=0}^{n}x^{k^2}$. Calculer $P_n$ en explicitant les propriétés de calcul utilisées. 
\end{exercise}

\begin{exercise}[subtitle= Lien produit-somme et  propriétés des produits, difficulty=**]
	Soit le produit $P_n = \prod\limits_{k=1}^{n}2^{1-k^3}$. Calculer $P_n$ en explicitant les propriétés de calcul utilisées. 
\end{exercise}


\begin{exercise}[subtitle= Binôme et somme, difficulty=*]
	Soit $p$ un entier naturel différent de 0. Calculer  $S_n = \sum\limits_{k=0}^{n}\binom{n}{k}p^{k}$.  
\end{exercise}

\begin{exercise}[subtitle= Binôme - sommes - récurrences, difficulty=$\spadesuit\spadesuit$]
Démontrer que : $\forall n \in \NNs$ :
$$\sum\limits_{k=1}^{n}\frac{(-1)^{k+1}}{k}\binom{n}{k} = \sum\limits_{k=1}^n \frac{1}{k}$$ 
Conseils :
\begin{tasks} 
	\task Démonstration par récurrence. 
	\task Utiliser la formule de Pascal. 
	\task Faire apparaître l'hypothèse de la démonstration, l'autre partie va devoir être égale à $\frac{1}{n+1}$.
	\task $\binom{n}{n+1}=0$ car on ne peut pas choisir $n+1$ éléments parmi $n$.
	\task $\frac{1}{k}\binom{n}{k-1} = \frac{1}{n+1}\binom{n+1}{k} $
	\task $\sum\limits_{k=0}^{n+1} (-1)^{k}\binom{n+1}{k} = 0$ 
\end{tasks} 
\end{exercise}


\begin{exercise}[subtitle= Formule de Van Der Monde, difficulty=$\spadesuit\spadesuit$]
	Démontrer que : $\forall n, m, q \in \NN$ :
	$$\sum\limits_{k=0}^{p}\binom{n}{k} \binom{m}{p-k} = \binom{n+m}{p}$$.  Dans un second temps, appliquer la formule aux sommes $S_n = \sum\limits_{k=0}^n \binom{n}{k}^2$ et $T_n = \sum\limits_{k=0}^n \binom{n}{k}^2$.
	Conseils :
	\begin{tasks} 
		\task Démonstration par récurrence. 
		\task $n$ est l'indice sur lequel porte la récurrence. 
		\task Initialisation avec $n=0$. Se rappeler que $\binom{p}{q} = 0 $ si $q>p$.
		\task Pour l'hérédité, appliquer une première fois la formule de Pascal. 
		\task Faire un changement d'indice $j=k-1$ et remarquer que pour $k=0$ le terme est nul. 
		\task Appliquer l'hypothèse de récurrence à $p$ et $p+1$.
		\task Appliquer une deuxième fois la formule de Pascal. 
	\end{tasks} 
\end{exercise}


\begin{exercise}[subtitle= Produit et téléscopages, difficulty=$\spadesuit$]
	Démontrer que : $P_n =\prod\limits_{k=2}^n \frac{k^3-1}{k^3+1} = \frac{2}{3}\frac{n^2 +n +1}{n(n+1)}$.
	Conseils :
	\begin{tasks} 
		\task Réécrire $k^3-1$ et $k^3+1$ en mettant $k-1$ et $k+1$ en facteur.
		\task Simplifier par téléscopage.
		\task Réécrire $k^2+k+1$  et $k^2-k+1$ en faisant apparaître $(k+1)^2$ et $(k-1)^2$. 
		\task Simplifer par téléscopage.  
	\end{tasks} 
\end{exercise}



