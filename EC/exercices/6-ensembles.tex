 \chapter{Ensembles et raisonnements}



\section{Questions de cours}	




\section{Exercices}


\begin{exercise}[subtitle= Absurde, difficulty=*]
	\begin{tasks} 
		\task 	Montrer que $\sqrt{2}$  n'est pas un nombre rationnel.
		\task Montrer qu'il n'existe pas d'entier naturel supérieur à tous les autres. 
		\task Montrer que l'ensemble des nombres premiers est infini.   
	\end{tasks}
\end{exercise}

\begin{exercise}[subtitle= Contraposition, difficulty=*]
		\begin{tasks} 
			\task Montrer que si $n$ est impair alors $n+1$ l'est aussi (directement et par contraposition).
			\task Montrer que si $3n$ est impair, alors $n$ est impair.  
			\task Montrer que si le produit de deux entiers $nm$ est impar, alors $n$ et $m$ sont impairs. 
			\task Montrer que $(\forall \epsilon > 0, \abs{a} < \epsilon) \Rightarrow a=0 )$
			\task Soit $n_1, n_2, \dots, n_9 \in \NN, \sum\limits_{i=1}^9 n_i =90$. Montrer qu'il existe 3 entiers $n_i$ dont la somme est supérieur ou égale à 30. On supposera que ces entiers sont ordonnés par leur indice (i.e. $n_1 \le n_2 \le \dots \le n_9$). 
		\end{tasks}
\end{exercise}



\begin{exercise}[subtitle= Disjonction des cas, difficulty=*]
	\begin{tasks} 
		\task Montrer que $\forall x,y \in \RR,  \abs{xy} =\abs{x}\abs{y}$.
		\task   
	\end{tasks}
\end{exercise}
