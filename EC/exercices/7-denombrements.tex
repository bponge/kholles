 \chapter{Dénombrement}
 
\section{Programme}
	\begin{itemize}
 	\item Cardinal, permutations, combinaisons
 	\item 
 \end{itemize}



\section{Exercices}




\begin{exercise}[subtitle= Dé à 20 faces]
	On dispose de 3 dés identiques à vingt faces et on les lance simultanément dans le but de disposer de 3 notes de colle. 
	\begin{tasks}
		\task Combien y-a-t-il de lancers possibles ?
		\task Combien y-a-t-il de lancers possibles tels que tous les dés présentent une valeur supérieure ou égale à 10 ?
		\task Combien y-a-t-il de lancers possibles tels que deux dés exactement présentent une valeur identique ? 
		\task Combien y-a-t-il de lancers possibles tels que les dés présentent  les valeurs 17, 9 et 3 ?
		\task Déduire les probabilités associées aux évènements précédents.  
		\task Soit $p$ la probabilité de l'évènement $\Ac$ "tous les dés présentent une valeur supérieure ou égale à 10". Que vaut la probabilité qu'au moins un dé présente une valeur strictement inférieur à 10 ?   
		\task Combien de lancers de 3 dés faudrait-t-il réaliser pour avoir 9 chances sur 10 d'obtenir un évènement $\Ac$ (i.e. que tous les élèves aient la moyenne$\dots$) ? 
	\end{tasks}
\end{exercise}


