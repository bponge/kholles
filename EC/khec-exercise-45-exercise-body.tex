% ------------------------------------------------------------------------
% file `khec-exercise-45-exercise-body.tex'
%
%     exercise of type `exercise' with id `45'
%
% generated by the `exercise' environment of the
%   `xsim' package v0.11 (2018/02/12)
% from source `khec' on 2020/11/17 on line 102
% ------------------------------------------------------------------------
	Démontrer que : $\forall n, m, q \in \NN$ :
	$$\sum\limits_{k=0}^{p}\binom{n}{k} \binom{m}{p-k} = \binom{n+m}{p}$$.  Dans un second temps, appliquer la formule aux sommes $S_n = \sum\limits_{k=0}^n \binom{n}{k}^2$ et $T_n = \sum\limits_{k=0}^n \binom{n}{k}^2$.
	Conseils :
	\begin{tasks}
		\task Démonstration par récurrence.
		\task $n$ est l'indice sur lequel porte la récurrence.
		\task Initialisation avec $n=0$. Se rappeler que $\binom{p}{q} = 0 $ si $q>p$.
		\task Pour l'hérédité, appliquer une première fois la formule de Pascal.
		\task Faire un changement d'indice $j=k-1$ et remarquer que pour $k=0$ le terme est nul.
		\task Appliquer l'hypothèse de récurrence à $p$ et $p+1$.
		\task Appliquer une deuxième fois la formule de Pascal.
	\end{tasks}
