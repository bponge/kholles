% ------------------------------------------------------------------------
% file `khec-exercise-57-exercise-body.tex'
%
%     exercise of type `exercise' with id `57'
%
% generated by the `exercise' environment of the
%   `xsim' package v0.19a (2020/03/19)
% from source `khec' on 2020/11/17 on line 75
% ------------------------------------------------------------------------
Pour les fonctions $f$ suivantes, on suivra le protocole standard pour étudier une fonction, à savoir :
\begin{enumerate}
	\item Déterminer l'ensemble de définition $\Dc_f$.
	\item Déterminer les symétries éventuelles de $f$ : parité et périodicité.
	\item Déterminer la dérivabilité de $f$,  zéros et infinis de $f$ pour les tangentes.
	\item Compléter un tableau de variation de $f$ qui comporte : les bornes de $\Dc_f$, les coupures dans $\Dc_f$ où $f$ n'est pas définie, les points clefs de $f'$, le signe de $f'$ et les variations de $f$ avec les valeurs limites au bout des flèches.
	\item Déterminer quelques points marquant de $f$, les zéros par exemple.
	\item Étudier les asymptotes de $f$.
	\item Tracer l'allure du graphe de $f$ avec tous les éléments.
\end{enumerate}
	\begin{tasks}(2)
		\task $f(x)= \frac{1}{1-e^x}$
	    \task $f(x)= \frac{e^x}{1-e^x}$
		\task $f(x)= \frac{1}{\ln(1+x)}$
		\task $f(x)= e^{\sqrt{x+1}}$
	    \task $f(x)= \ln{\sqrt{x+1}}$
	    \task $f(x)= \sin(3 \pi x)$
	    \task $f(x)= \frac{1}{\frac{1}{2}-\cos(\pi x)}$
	    \task $f(x)= \frac{cos(\pi x)}{\frac{1}{2}-\cos(\pi x)}$
	\end{tasks}
