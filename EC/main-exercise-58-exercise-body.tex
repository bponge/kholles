% ------------------------------------------------------------------------
% file `main-exercise-58-exercise-body.tex'
%
%     exercise of type `exercise' with id `58'
%
% generated by the `exercise' environment of the
%   `xsim' package v0.11 (2018/02/12)
% from source `main' on 2020/11/17 on line 21
% ------------------------------------------------------------------------
		\begin{tasks}
			\task Montrer que si $n$ est impair alors $n+1$ l'est aussi (directement et par contraposition).
			\task Montrer que si $3n$ est impair, alors $n$ est impair.
			\task Montrer que si le produit de deux entiers $nm$ est impar, alors $n$ et $m$ sont impairs.
			\task Montrer que $(\forall \epsilon > 0, \abs{a} < \epsilon) \Rightarrow a=0 )$
			\task Soit $n_1, n_2, \dots, n_9 \in \NN, \sum\limits_{i=1}^9 n_i =90$. Montrer qu'il existe 3 entiers $n_i$ dont la somme est supérieur ou égale à 30. On supposera que ces entiers sont ordonnés par leur indice (i.e. $n_1 \le n_2 \le \dots \le n_9$).
		\end{tasks}
