% ------------------------------------------------------------------------
% file `main-exercise-5-exercise-body.tex'
%
%     exercise of type `exercise' with id `5'
%
% generated by the `exercise' environment of the
%   `xsim' package v0.11 (2018/02/12)
% from source `main' on 2020/11/17 on line 66
% ------------------------------------------------------------------------
	Soit $a$,$b$ et $c$ trois réels et $a>0$. On s'intéresse au signe de $ax^2+bx + c=a(x-x_1)(x-x_2)$, avec $x_1<x_2$.
	\begin{tasks}
		\task Discuter le signe de $ax^2+bx + c$ en fonction de la valeur de  $x$ lorsque $x$ est réel.
		\task Peut-on généraliser une règle pour le signe ?
	\end{tasks}
