%% GENERAL PACKAGE AND LAYOUT
\usepackage[T1]{fontenc}
\usepackage[utf8]{inputenc}
\usepackage{lmodern}
\usepackage[french]{babel}
\usepackage[a4paper,margin=1.5cm]{geometry}
\usepackage[hidelinks]{hyperref}
\hypersetup{
	colorlinks,
	linkcolor={blue!50!black},
	citecolor={blue!50!black},
	urlcolor={blue!80!black}
}
\usepackage[dvipsnames,svgnames]{xcolor}
\usepackage{listings}   
\usepackage{xspace}

%% MATH
\usepackage{algorithm}
\usepackage{algorithmicx}
\usepackage[noend]{algpseudocode}
\usepackage{amsmath, amssymb, amsfonts}
\usepackage[ntheorem]{empheq} 
\usepackage[thmmarks,amsmath]{ntheorem}
%\usepackage{mathcommand}


%% STYLE
\usepackage{fancyhdr}
\setlength{\headheight}{15.2pt}
\pagestyle{headings}


\usepackage{titlesec}

\titleformat{\chapter}[display]
{\normalfont\Large\filcenter\sffamily}
{\titlerule[1pt]%
	\vspace{1pt}%
	\titlerule
	\vspace{1pc}%
	\LARGE\MakeUppercase{\chaptertitlename} \thechapter}
{1pc}
{\titlerule
	\vspace{1pc}%
	\Huge}

\titleformat{\section}[frame]
{\normalfont}
{\filright
	\footnotesize
	\enspace SECTION \thesection\enspace}
{8pt}
{\Large\bfseries\filcenter}


%% INCLUDE MATH RELATED SETTINGS

% OR

\newcommand{\NN}{\mathbb{N}}
\newcommand{\ZZ}{\mathbb{Z}}
\newcommand{\QQ}{\mathbb{Q}}
\newcommand{\RR}{\mathbb{R}}
\newcommand{\CC}{\mathbb{C}}
\newcommand{\KK}{\mathbb{K}}
\newcommand{\PP}{\mathbb{P}}
\newcommand{\dt}{\,\mathrm d}
\newcommand{\transp}{\,{}^t\!}
\newcommand{\fnc}[5]{\begin{align*}#1:#2&\longrightarrow#3\\#4&\longmapsto#5\end{align*}}


\newcommand{\NNs}{\mathbb{N}^\star}
\newcommand{\ZZs}{\mathbb{Z}^\star}
\newcommand{\QQs}{\mathbb{Q}^\star}
\newcommand{\RRs}{\mathbb{R}^\star}
\newcommand{\CCs}{\mathbb{C}^\star}
\newcommand{\KKs}{\mathbb{K}^\star}

\newcommand{\Sc}{\mathcal{S}}
\newcommand{\Ec}{\mathcal{E}}
\newcommand{\Dc}{\mathcal{D}}
\newcommand{\Pc}{\mathcal{P}}
\newcommand{\Qc}{\mathcal{Q}}
\newcommand{\Ac}{\mathcal{A}}
\newcommand{\Bc}{\mathcal{B}}
\newcommand{\Cc}{\mathcal{C}}
\newcommand{\Ic}{\mathcal{I}}
\newcommand{\Oc}{\mathcal{O}}

\newcommand{\PGCD}{\mathrm{PGCD}}
\newcommand{\abs}[1]{\mid #1 \mid}


\renewmathcommand{\t}{\expandafter\mathrel\expandafter\text}





% EM


% Racourcis Mathematiques
\newcommand{\C}			{\mathbb{C}}
\newcommand{\R}			{\mathbb{R}}
\newcommand{\N}			{\mathbb{N}}
\renewcommand{\P}		{\mathbb{P}}
\newcommand{\Q}			{\mathbb{Q}}
\newcommand{\Z}			{\mathbb{Z}}
\newcommand{\K}			{\mathbb{K}}
\newcommand{\U}			{\mathbb{U}}

\newcommand{\ov}[1]		{\overline{#1}}
\newcommand{\ti}[1]		{\widetilde{#1}}
\newcommand{\ha}[1]		{\widehat{#1}}
\newcommand{\ca}[1]		{\mathcal{#1}}

\newcommand{\Coo}		{\mathcal{C}^\infty}



%quotients
\def\quot#1#2{\mathchoice
	{\vbox{\hbox{$\displaystyle #1$}\kern -.5ex}%
		\over
		\vbox{\kern 0pt\hbox{$\displaystyle #2$}}%
	}%
	{\vbox{\hbox{$\displaystyle #1$}\kern .25ex}%
		\over
		\vbox{\kern .5ex\hbox{$\displaystyle #2$}}%
	}%
	{#1/#2}%
	{#1/#2}%
}

% fonctions hyperboliques

\def\th{\mathop{\rm th}\nolimits}
\def\ch{\mathop{\rm ch}\nolimits}
\def\sh{\mathop{\rm sh}\nolimits}
\def\argsh{\mathop{\rm argsh}\nolimits}
\def\argch{\mathop{\rm argch}\nolimits}
\def\argth{\mathop{\rm argth}\nolimits}

% algèbre linéaire
\def\id{\mathop{\rm id }\nolimits}
\def\Im{\mathop{\rm Im }\nolimits}
\def\Vect{\mathop{\rm Vect }\nolimits}

% géométrie
\def\flec
{\overrightarrow}

\def\lam{\lambda}
\def\eps{\varepsilon}

% nombres complexes
\def\Re{\mathop{\rm Re}\nolimits}
% - somme de #1=#2 \`a #3
\def \Som#1#2#3{{\displaystyle \sum_{#1=#2}^{#3}}\, }
%--% - somme pour #1
\def \SOM#1{{\displaystyle \sum_{#1}}\, }
\newcommand{\dsp}{\displaystyle}

\usepackage{enumitem}
\setlist[enumerate,1]{label={\upshape(\roman*)}}

\newcounter{csection}
\numberwithin{csection}{section}

%\counterwithin*{thm}{chapter}

\theoremstyle{break} \theoremheaderfont{\normalfont\bfseries}\theorembodyfont{\slshape} \theoremsymbol{\ensuremath{\bigstar}}
\theoremindent1cm 
\theoremseparator{} 
%\theoremprework{\medskip} 
%\theorempostwork{\smallskip} 
\newtheorem{Theorem}[csection]{Théorème}

\theoremstyle{plain}
\theoremsymbol{} 
\theoremindent1cm 
\theoremnumbering{greek} 
\newtheorem{Lemma}[csection]{Lemme}

\theoremindent1cm 
\theoremsymbol{} 
\theoremnumbering{arabic} 
\newtheorem{Corollary}[csection]{Corollaire}

\theoremindent1cm 
\theoremsymbol{} 
\theoremnumbering{arabic} 
\newtheorem{Proposition}[csection]{Proposition}

\theoremstyle{break} 
\theorembodyfont{\upshape} 
\theoremsymbol{\ensuremath{\ast}} 
\theoremseparator{} 
\newtheorem{Example}[csection]{Exemple}

\theoremstyle{plain} 
\theoremsymbol{\ensuremath{\vartriangle}} 
\theoremseparator{ --- } 
\theoremindent1cm 
%\theoremprework{\smallskip} 
%\theorempostwork{\smallskip} 
\newtheorem{Definition}[csection]{Définition}

\theoremheaderfont{\sc}
\theorembodyfont{\upshape} 
\theoremstyle{nonumberplain}
\theoremseparator{ --- } 
\theoremindent2cm 
\theoremsymbol{\rule{1ex}{1ex}} 
\newtheorem{Proof}{Démonstration}

\theoremstyle{break} 
\theoremheaderfont{\sc\bfseries}
\theorembodyfont{} 
\theoremseparator{ --- } 
\theoremindent1cm 
%\theoremsymbol{\rule{1ex}{1ex}} 
%\theoremnumbering{greek} 
\newtheorem{Trick}[csection]{Formule}


\theoremstyle{break} 
\theoremheaderfont{\sc\bfseries}
\theorembodyfont{} 
\theoremseparator{ --- } 
\theoremindent1cm 
%\theoremsymbol{\rule{1ex}{1ex}} 
%\theoremnumbering{greek} 
\newtheorem{Method}[csection]{Méthode}


\theoremstyle{break} 
\theoremheaderfont{\sc\bfseries}
\theorembodyfont{} 
\theoremseparator{ --- } 
\theoremindent1cm 
%\theoremsymbol{\rule{1ex}{1ex}} 
%\theoremnumbering{greek} 
\newtheorem{Question}[csection]{Question}

%\theoremstyle{plain} 
%\theoremheaderfont{\sc\bfseries}
%\theorembodyfont{} 
%\theoremseparator{ --- } 
%\theoremindent1cm 
%\theoremsymbol{\bfseries\textquestiondown} 
%%\theoremnumbering{greek} 
%\newtheorem{Question}[csection]{Question}
\usepackage{xcolor}
\usepackage{xsim,tasks,siunitx}
\xsimsetup{
	%clear-aux,
	solution/print = false
%	no-files = true
}

\DeclareExerciseTagging{difficulty}
%\xsimsetup{
%	difficulty={*, **, ***}
%}
\DeclareExerciseTagging{extype}
%\xsimsetup{
%	extype={cours, résolution, démonstration, calcul}
%}


\DeclareExerciseEnvironmentTemplate{custom}
{
	\subsection*
	{%
		\XSIMmixedcase {\GetExerciseName}\nobreakspace
		\GetExerciseProperty{counter}%
		\IfExercisePropertySetT{difficulty}
		{ {\normalfont\bfseries \GetExerciseProperty{difficulty}}}%
		\IfExercisePropertySetT{subtitle}
		{ {\normalfont\GetExerciseProperty{subtitle}}}%
		\IfExercisePropertySetT{extype}
		{ {\normalfont\itshape (\GetExerciseProperty{extype})}}%
	}%
}
{}

\xsimsetup{exercise/template = custom}






\newcommand{\mq}{montrer que }
\newcommand{\Mq}{Montrer que }
\newcommand\NB[1][0.3]{N\kern-#1em\textcolor{red}{B} ---  }

\newcommand*{\eg}{e.g.\@\xspace}
\newcommand*{\ie}{i.e.\@\xspace}

\makeatletter
\newcommand*{\etc}{%
	\@ifnextchar{.}%
	{etc}%
	{etc.\@\xspace}%
}
\makeatother

\setcounter{tocdepth}{3}     
\setcounter{secnumdepth}{3}  


